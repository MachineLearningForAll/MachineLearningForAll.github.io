\documentclass{article}
\usepackage{graphicx}
\usepackage{caption}
\usepackage{amsmath, amssymb}
\usepackage[margin=2.5cm]{geometry}

\begin{document}

\section*{Markovsinequality}

Consider a real-valued non-negative random variable (RV) $x$ for which
 the expectation $\mathbb{E} \{ x\}$ exists. \index{Markov's inequality}
 Markov's inequality provides an upper bound on the probability
 $\mathbb{P}\left(x\geq a\right)$ that $x$ exceeds a given positive threshold $a>0$.
 In particular,
 \begin{equation}
 \mathbb{P}\left(x \geq a\right) \leq \frac{\mathbb{E} \{ x\}}{a} \qquad \mbox{ holds for any } a > 0.
 \label{eq:markovsinequality_dict}
 \end{equation}
 This inequality can be verified by noting that
 $\mathbb{P}\left(x \geq a\right)$ is the expectation $\mathbb{E} {g(x)}$ with
 the function
 $$g: \mathbb{R} \rightarrow \mathbb{R}: x' \mapsto \mathbb{I}_{\{x \geq a\}}(x').$$
 As illustrated in Figure \ref{fig:markovsinequality_dict}, for any positive $a>0$,
 $$ g(x') \leq x'/a \mbox{ for all } x' \in \mathbb{R}.$$ This implies \eqref{eq:markovsinequality_dict}
 via the monotonicity of the Lebesgue integral \cite[p. 50]{folland1999real}.
 \begin{figure}
 \centering
 \includegraphics[width=0.8\linewidth]{../images/markovsinequality_tikz.png}
 \caption{The expectation $\mathbb{E} \{x\}$ and the probability $\mathbb{P}\left(x \geq a\right)$
 of a non-negative random variable (RV) with probability distribution $P^{(x)}$
 can be obtained via Lebesgue integral of
 $f(x') = x'/a$ and $g(x') = \mathbb{I}_{\{x \geq a\}}(x')$, respectively.}
 \label{fig:markovsinequality_dict}
 \end{figure} \\
 See also: expectation, probability, concentration inequality.

\end{document}
